\documentclass[xcolor=dvipsnames]{beamer}

\usetheme{Madrid}
\useoutertheme{miniframes} % Alternatively: miniframes, infolines, split
\useinnertheme{circles}

\definecolor{IITHorange}{RGB}{243, 130, 33} % UBC Blue (primary)
\definecolor{IITHyellow}{RGB}{254, 203, 10} % UBC Grey (secondary)

\setbeamercolor{palette primary}{bg=IITHorange,fg=white}
\setbeamercolor{palette secondary}{bg=IITHorange,fg=white}
\setbeamercolor{palette tertiary}{bg=IITHorange,fg=white}
\setbeamercolor{palette quaternary}{bg=IITHorange,fg=white}
\setbeamercolor{structure}{fg=IITHorange} % itemize, enumerate, etc
\setbeamercolor{section in toc}{fg=IITHorange} % TOC sections

% Override palette coloring with secondary
\setbeamercolor{subsection in head/foot}{bg=IITHyellow,fg=white}

\setbeamertemplate{caption}[numbered]

\title[Cryptanalysis of DES]{CS5760: Cryptanalysis of DES and DES-like Iterated Cryptosystems}
\date{February 3, 2025}
\author{Gautam Singh}
\institute[IITH]{Indian Institute of Technology Hyderabad}

\begin{document}
	
	\begin{frame}
		\titlepage
	\end{frame}
	
	\begin{frame}
		\tableofcontents
	\end{frame}
	
	\section{Introduction to Differential Cryptanalysis}

	\subsection{Differential Cryptanalysis}
	
	\begin{frame}
		\frametitle{Differential Cryptanalysis}
		\begin{columns}
			\begin{column}{0.49\textwidth}
				\begin{enumerate}
					\item<1-> Chosen plaintext attack.
					\item<1-> Exploit XOR between plaintext pairs to find key bits.
					\item<2-> Per DES round, XOR is invariant under:
					\begin{itemize}
						\item Expansion \(E\) to get \(S_E\).
						\item Key mixing with subkey \(S_K\) to get \(S_I = S_E
						\oplus S_K\).
						\item Permutation on \(S_O\) after S boxes.
					\end{itemize}
					\item<3-> S boxes are \emph{nonlinear}. Probability analysis
					performed on XOR of S box inputs and outputs.
				\end{enumerate}
			\end{column}
			\begin{column}{0.49\textwidth}
				\begin{figure}[!ht]
					\centering
					\includegraphics<2->[width=\columnwidth]{images/des_f.png}
					\only<2->{\caption{\(F\) function of DES.}}
					\label{fig:des-f}
				\end{figure}
			\end{column}
		\end{columns}
	\end{frame}

\end{document}