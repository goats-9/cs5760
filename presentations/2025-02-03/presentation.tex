% Set the document class and theme
\documentclass{beamer}

\usetheme{Madrid}
\useoutertheme{miniframes} % Alternatively: miniframes, infolines, split
\useinnertheme{circles}

\definecolor{IITHorange}{RGB}{243, 130, 33} % UBC Blue (primary)
\definecolor{IITHyellow}{RGB}{254, 203, 10} % UBC Grey (secondary)

\setbeamercolor{palette primary}{bg=IITHorange,fg=white}
\setbeamercolor{palette secondary}{bg=IITHorange,fg=white}
\setbeamercolor{palette tertiary}{bg=IITHorange,fg=white}
\setbeamercolor{palette quaternary}{bg=IITHorange,fg=white}
\setbeamercolor{structure}{fg=IITHorange} % itemize, enumerate, etc
\setbeamercolor{section in toc}{fg=IITHorange} % TOC sections

% Override palette coloring with secondary
\setbeamercolor{subsection in head/foot}{bg=IITHyellow,fg=white}

\setbeamertemplate{caption}[numbered]

\usepackage{./presentation_macros}

\title[Cryptanalysis of DES]{CS5760: Cryptanalysis of DES and DES-like Iterated Cryptosystems}
\date{February 3, 2025}
\author{Gautam Singh}
\institute[IITH]{Indian Institute of Technology Hyderabad}

\begin{document}
	
	\begin{frame}
		\titlepage
	\end{frame}
	
	\begin{frame}
		\tableofcontents
	\end{frame}
	
	\section{Differential Cryptanalysis}
	
	\begin{frame}
		\frametitle{Differential Cryptanalysis}
		\begin{columns}
			\begin{column}{0.65\textwidth}
				\begin{enumerate}
					\item<1-> Chosen plaintext attack.
					\item<1-> Exploit XOR between plaintext pairs to find key bits.
					\item<2-> Per DES round, XOR of respective inputs is:
					\begin{itemize}
						\item \emph{Linear} in expansion \(E\) to get \(S_E\).
						\item \emph{Invariant} in key mixing with subkey \(S_K\)
						to get \(S_I = S_E \oplus S_K\).
						\item \emph{Linear} in permutation \(P\) on \(S_O\)
						after S boxes.
						\item \emph{Invariant} in XOR operation connecting
						rounds.
					\end{itemize}
					\item<3-> S boxes are \emph{nonlinear}. Probability analysis
					performed between input and output XOR.
				\end{enumerate}
			\end{column}
			\begin{column}{0.35\textwidth}
				\begin{figure}[!ht]
					\centering
					\includegraphics<2->[width=\columnwidth]{images/des_f.png}
					\only<2->{\caption{\(F\) function of DES.}}
					\label{fig:des-f}
				\end{figure}
			\end{column}
		\end{columns}
	\end{frame}

	\section{Probability Analysis of S Boxes}
	
	\begin{frame}
		\frametitle{Probability Analysis of S Boxes}
		\begin{enumerate}
			\item Suppose \(Si_I^\prime = Si_I \oplus Si_I^*\) is the input XOR
			to the \(i\)-th S box, and \(Si_O^\prime\) is the output XOR (\(1
			\le i \le 8\)).
			\item<2-> We create a \emph{pairs XOR distribution table} for each S
			box.
			\begin{itemize}
				\item Each entry \(\brak{Si_I^\prime, Si_O^\prime}\) equals the
				number of 6-bit key blocks \(Si_K\) for which \(Si_I^\prime
				\rightarrow Si_O^\prime\).
				\item 64-by-16 joint probability mass function.
			\end{itemize}
			\item<3-> This joint PMF can reduce the number of possible
			(sub)keys. Used to drive choice for the plaintext XOR.
			\begin{itemize}
				\item \(\approx 80\%\) entries are non-zero/possible for each S
				box (some have lesser percentages).
				\item Given \(Si_I^\prime\) and \(Si_O^\prime\), we can narrow
				down \(Si_K\) to a few possibilities.
			\end{itemize}
			\item<4-> \(i\)-th S box contributes probability \(p_i\) for
			\(Si_I^\prime \rightarrow Si_O^\prime\). 
			\begin{itemize}
				\item For \(X \rightarrow Y\) over a round, \(P = \prod_i p_i\).
				\item Over \(n\) rounds, \(P = \prod_{i=1}^n P_i\).
			\end{itemize}
			\only<5->{\bfseries Desirable for cryptanalysis: high \(P\) with
			large \(n\).}
		\end{enumerate}
	\end{frame}

	\section{Characteristic}

	\begin{frame}
		\frametitle{Characteristic}
		Formalizes notion of high-probability plaintext XORs.
		\begin{definition}[Characteristic]
			An \emph{n-round chracteristic} is a tuple \(\Omega =
			\brak{\Omega_P, \Omega_\Lambda, \Omega_T}\) where \(\Omega_P =
			\brak{L^\prime, R^\prime}\) and \(\Omega_T = \brak{l^\prime,
			r^\prime}\) are \(m\) bit numbers, \(\Omega_\Lambda =
			\brak{\Lambda_1, \ldots, \Lambda_n}\), \(\Lambda_i =
			\brak{\lambda_I^i, \lambda_O^i}\) and \(\lambda_I^i,
			\lambda_O^i, L^\prime, R^\prime, l^\prime, r^\prime\) are
			\(\frac{m}{2}\) bit numbers and \(m\) is the block size of the
			cryptosystem satisfying
			\begin{align}
				\lambda_I^1 &= R^\prime \\
				\lambda_I^2 &= L^\prime \oplus \lambda_O^1 \\
				\lambda_I^n &= r^\prime \\
				\lambda_I^{n-1} &= l^\prime \oplus \lambda_O^n \\
				\forall\ 1 < i < n,\ \lambda_O^i &= \lambda_I^{i-1} \oplus \lambda_I^{i+1}
				\label{eq:char-def}
			\end{align}
		\end{definition}
	\end{frame}

\end{document}