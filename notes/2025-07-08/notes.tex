% To familiarize yourself with this template, the body contains
% some examples of its use.  Look them over.  Then you can
% run LaTeX on this file.  After you have LaTeXed this file then
% you can look over the result either by printing it out with
% dvips or using xdvi.
%

\documentclass[twoside]{article}
%\usepackage{soul}
\usepackage{./lecnotes_macros}


\begin{document}
%FILL IN THE RIGHT INFO.
%\lecture{**LECTURE-NUMBER**}{**DATE**}{**LECTURERS**}{**SCRIBE**}
\lecture{11}{Gr\"{o}bner Bases over Rings}{Maria Francis}{Gautam Singh}{7 July 2025}
%\footnotetext{These notes are partially based on those of Nigel Mansell.}

%All figures are to be placed in a separate folder named ``images''

% **** YOUR NOTES GO HERE:

\section{Gr\"{o}bner Bases over Principal Ideal Domains (PIDs)}
\label{sec:groebner-bases-over-pids}

Here, we consider the univariate polynomial ring \(\bZ\brak{x}\). The ordering
in \(\bZ\) is \(a_1 < a_2\) if \(\abs{a_1} < \abs{a_2}\) or if \(\abs{a_1} =
\abs{a_2}\) and \(a_1\) is negative.

\begin{definition}[Reduction]
    We say that \(f \xrightarrow{g} h\) if \(\mathrm{lm}\brak{g} \mid
    \mathrm{lm}\brak{f}\) and \(\exists a, b \in \bZ\) such that
    \(\mathrm{lc}\brak{f} = a\mathrm{lc}\brak{g} + b\) where \(a \neq 0\) and
    \(b < \mathrm{lc}\brak{f}\).
\end{definition}

\section{Strong and Weak Gr\"{o}bner Basis}

\begin{definition}[Strong Gr\"{o}bner Basis]
    A set of polynomials \(G = \brak{g_1, \ldots, g_t}\) is a \emph{strong
    Gr\"{o}bner basis} for an ideal \(I\) if for any \(f \in I \setminus
    \cbrak{0}\), \(\exists a, g \in G\) such that \(\mathrm{lt}\brak{g} \mid
    \mathrm{lt}\brak{f}\).
\end{definition}

To construct a strong Gr\"{o}bner basis, we required to construct a G-polynomial
in addition to an S-polynomial.

\begin{definition}[S/G-Polynomial]
    Let \(f, g \in R\sbrak{x}\). WLOG let \(\mathrm{lc}\brak{f} <
    \mathrm{lc}\brak{g}\). Let \(t = \mathrm{lcm}\brak{\mathrm{lm}\brak{f},
    \mathrm{lm}\brak{g}}\). Define
    \begin{equation}
        t_f = \frac{t}{\mathrm{lm}\brak{f}}, \quad
        t_g = \frac{t}{\mathrm{lm}\brak{g}}.
    \end{equation}
    Similarly, let \(a = \mathrm{lcm}\brak{\mathrm{lc}\brak{f},
    \mathrm{lc}\brak{g}}\). Define
    \begin{equation}
        a_f = \frac{a}{\mathrm{lc}\brak{f}}, \quad
        a_g = \frac{a}{\mathrm{lc}\brak{g}}.
    \end{equation}
    Then, the S polynomial of \(f\) and \(g\) is defined as
    \begin{equation}
        S\brak{f, g} = a_ft_ff - a_gt_gg.
    \end{equation}
    Let \(b = \gcd\brak{\mathrm{lc}\brak{f}, \mathrm{lc}\brak{g}}\). Then, by
    the Extended Euclidean algorithm, we have \(b = b_f\mathrm{lc}\brak{f} +
    b_g\mathrm{lc}\brak{g}\) for some \(b_f, b_g\). Then, the G-polynomial of
    \(f\) and \(g\) is given by
    \begin{equation}
        G\brak{f, g} = b_ft_ff - b_gt_gg.
    \end{equation}
\end{definition}

\begin{definition}[S/G-Pairs]
    Let \(\cbrak{f_1, \ldots, f_m}\) be the set \(R^m\). Let \(\alpha, \beta \in
    R^m\). We assume that \(\mathrm{lc}\brak{\bar{\alpha}} <
    \mathrm{lc}\brak{\bar{\beta}}\). Let \(t =
    \mathrm{lcm}\brak{\mathrm{lm}\brak{\bar{\alpha}},
    \mathrm{lm}\brak{\bar{\beta}}}\). Define
    \begin{equation}
        t_\alpha = \frac{t}{\mathrm{lm}\brak{\bar{\alpha}}}, \quad
        t_\beta = \frac{t}{\mathrm{lm}\brak{\bar{\beta}}}.
    \end{equation}
    Let \(a = \mathrm{lcm}\brak{\mathrm{lc}\brak{\bar{\alpha}},
    \mathrm{lc}\brak{\bar{\beta}}}\). Define
    \begin{equation}
        a_\alpha = \frac{a}{\mathrm{lc}\brak{\bar{\alpha}}}, \quad
        a_\beta = \frac{a}{\mathrm{lc}\brak{\bar{\beta}}}.
    \end{equation}
    Then, the S-pair of \(\alpha\) and \(\beta\) is defined as
    \begin{equation}
        \mathrm{Spair}\brak{\alpha, \beta} = a_\alpha t_\alpha \alpha - a_\beta t_\beta \beta.
    \end{equation}
    Let \(b = \gcd\brak{\mathrm{lc}\brak{\bar{\alpha}},
    \mathrm{lc}\brak{\bar{\beta}}}\). Then, the G-pair of \(\alpha\) and
    \(\beta\) is defined as
    \begin{equation}
        \mathrm{Gpair}\brak{\alpha, \beta} = b t_\alpha \alpha - b t_\beta \beta.
    \end{equation}
\end{definition}

\begin{definition}[S-Reduction]
    Let \(\alpha, \beta \in R^m\). We say that \(\beta\) s-reduces to \(\alpha\)
    if \(\bar{\beta}\) reduces \(\bar{\alpha}\) and \(\mathrm{Sig}\brak{\alpha}
    > \mathrm{Sig}\brak{\beta}\) where \(\mathrm{lc}\brak{\bar{\alpha}} =
    a\mathrm{lc}\brak{\bar{\beta}} + b\) for some \(a, b \in R\) where \(a \neq
    0\) and \(b < \mathrm{lc}\brak{\bar{\alpha}}\).
\end{definition}

\end{document}
