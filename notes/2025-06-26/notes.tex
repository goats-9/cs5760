% To familiarize yourself with this template, the body contains
% some examples of its use.  Look them over.  Then you can
% run LaTeX on this file.  After you have LaTeXed this file then
% you can look over the result either by printing it out with
% dvips or using xdvi.
%

\documentclass[twoside]{article}
%\usepackage{soul}
\usepackage{./lecnotes_macros}


\begin{document}
%FILL IN THE RIGHT INFO.
%\lecture{**LECTURE-NUMBER**}{**DATE**}{**LECTURERS**}{**SCRIBE**}
\lecture{9}{Introduction to Gr\"{o}bner Bases}{Maria Francis}{Gautam Singh}{26 June 2025}
%\footnotetext{These notes are partially based on those of Nigel Mansell.}

%All figures are to be placed in a separate folder named ``images''

% **** YOUR NOTES GO HERE:

In multivariate polynomial rings, long division can yield different results
depending on the monomial order chosen. Gr\"{o}bner bases provide a systematic
way to handle these variations by defining a canonical form for polynomials. In
particular, the ordering of divisors does not matter in \(k\sbrak{x_1, \ldots,
x_n}\).

\section{Monomial Orders}
\label{sec:monomial-orders}

We begin by defining a \emph{monomial order} on the polynomial ring
\(k\sbrak{x_1, \ldots, x_n}\).

\begin{definition}[Monomial Order]
    A monomial order is an order on the set of monomials in \(k\sbrak{x_1,
    \ldots, x_n}\) satisfying the following properties:
    \begin{enumerate}
        \item It is a \emph{total ordering}.
        \item If \(\alpha > \beta\) are monomials, then \(\alpha\gamma >
        \beta\gamma\) for any monomial \(\gamma\).
        \item It is \emph{well ordering}, meaning every non-empty set of
        monomials has a least element.
    \end{enumerate}
\end{definition}

An example of a monomial order is \emph{lexicographic order} (lex), where we set
an ordering on the variables such as \(x_1 > x_2 > \ldots > x_n\) and compare
monomials by considering the leftmost nonzero element in their pointwise
difference. Accounting for the total degree of the monomials gives the
\emph{graded lexicographic order} (grlex) and \emph{graded reverse lexicographic
order} (grrevlex).

\section{Monomial Ideals}
\label{sec:monomial-ideals}

\begin{definition}[Monomial Ideal]
    An ideal \(I \subseteq k\sbrak{x_1, \ldots, x_n}\) is called a
    \emph{monomial ideal} if there is a finite subset \(A \subset \bN^n\) such
    that \(I\) consists of all polynomials that can be written as finite sums of
    monomials \(c x^\alpha\) where \(c \in k\) and \(\alpha \in A\).
\end{definition}

In other words, monomial ideals are those which have a generator solely
consisting of monomials.

\begin{lemma}[Dickson's Lemma]
    All monomial ideals in \(k\sbrak{x_1, \ldots, x_n}\) are finitely generated.
\end{lemma}

\begin{theorem}[Hilbert Basis Theorem]
    Any ideal in \(R\sbrak{x_1, \ldots, x_n}\) is finitely generated if and only
    if it is 
\end{theorem}

\begin{definition}
    Let \(I \in k\sbrak{x_1, \ldots, x_n}\) be an ideal other than
    \(\cbrak{0}\). We define the following sets.
    \begin{enumerate}
        \item \(\mathrm{LT}\brak{I}\) is the set of leading terms of the
        polynomials in \(I\).
    \end{enumerate}
\end{definition}

Suppose that \(I = \langle f_1, \ldots, f_s \rangle\) is an ideal. Then
\(\langle \mathrm{LT}\brak{f_1}, \ldots, \mathrm{LT}\brak{f_s} \rangle \subseteq
\mathrm{LT}\brak{I}\), with equality iff \(\langle f_1, \ldots, f_s \rangle\) is
a Gr\"{o}bner basis of \(I\).

\begin{proposition}[Existence of Gr\"{o}bner Bases]
    Let \(I \subseteq k\sbrak{x_1, \ldots, x_n}\) be an ideal. Then,
    \begin{enumerate}
        \item \(\langle \mathrm{LT}\brak{I} \rangle\) is a monomial ideal.
        \item \(\exists\ g_1, \ldots, g_t \in I\) such that
        \(\langle \mathrm{LT}\brak{g_1}, \ldots, \mathrm{LT}\brak{g_t} \rangle =
        \langle \mathrm{LT}\brak{I} \rangle\).
    \end{enumerate}
\end{proposition}

\begin{definition}[Gr\"{o}bner Basis]
    Fix a monomial order. A finite subset \(G = \cbrak{g_1, \ldots, g_t}\) of an
    ideal \(I\) is said for be a Gr\"{o}bner basis of \(I\) iff \(\langle
    \mathrm{LT}\brak{g_1}, \ldots, \mathrm{LT}\brak{g_t} \rangle = \langle
    \mathrm{LT}\brak{I} \rangle\).
\end{definition}
In particular, we also have \(\langle g_1, \ldots, g_t \rangle = I\).

\section{Properties of Gr\"{o}bner Bases}
\label{sec:properties-of-groebner-bases}

\begin{enumerate}
    \item Remainder on division of \(f\) by \(G\) is unique.
\end{enumerate}

\begin{definition}[S-Polynomial]
    Let \(f, g \in k\sbrak{x_1, \ldots, x_n}\) be two polynomials with
    \(\mathrm{lm}\brak{f} = x^\alpha\) and \(\mathrm{lm}\brak{g} = x^\beta\) and
    \(\gamma\) be the least common multiple (LCM) of \(\mathrm{lm}\brak{f}\) and
    \(\mathrm{lm}\brak{g}\), i.e., \(\gamma_i = \max\brak{\alpha_i, \beta_i}\).
    The S-polynomial of \(f\) and \(g\) is defined as
    \begin{equation}
        S\brak{f, g} \triangleq x^\gamma \brak*{\frac{f}{\mathrm{lt}\brak{f}} - \frac{g}{\mathrm{lt}\brak{g}}}
        \label{eq:s-polynomial}
    \end{equation}
\end{definition}
The S-polynomial is a way to combine two polynomials such that their leading
terms are eliminated.

\begin{theorem}[Buchberger's Criterion]
    Let \(I\) be a polynomial ideal. Then a basis \(G = \cbrak{g_1, \ldots,
    g_t}\) of \(I\) is a Gr\"{o}bner basis of \(I\) iff for all \(f, g \in I, f
    \neq g\), the S-polynomial \(S\brak{f, g}\) reduces to zero modulo \(G\).
\end{theorem}
This gives us \emph{Buchberger's algorithm} for computing Gr\"{o}bner bases.

\section{Buchberger's Algorithm}

\textbf{Input:} Finite set of polynomials \(F = \cbrak{f_1, \ldots, f_s}\).

\textbf{Output:} Gr\"{o}bner basis \(G\) of the ideal generated by \(F\).

\begin{enumerate}
    \item Set \(G \gets F\).
    \item For each pair of polynomials \(f_i, f_j \in G\):
    \begin{enumerate}
        \item Compute the S-polynomial \(S\brak{f_i, f_j}\) using
        \eqref{eq:s-polynomial}.
        \item Reduce \(S\brak{f_i, f_j}\) modulo \(G\).
        \item If the result is non-zero, add it to \(G\).
    \end{enumerate}
    \item Repeat step 2 until no new polynomials are added to \(G\).
\end{enumerate}

To speed up this algorithm, especially the reduction step, we can use signatures
of polynomials \(p\) to predict which polynomials in \(G\) will contribute to
the reduction of \(p\).
\end{document}
