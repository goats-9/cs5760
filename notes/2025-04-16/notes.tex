% To familiarize yourself with this template, the body contains
% some examples of its use.  Look them over.  Then you can
% run LaTeX on this file.  After you have LaTeXed this file then
% you can look over the result either by printing it out with
% dvips or using xdvi.
%

\documentclass[twoside]{article}
%\usepackage{soul}
\usepackage{./lecnotes_macros}


\begin{document}
%FILL IN THE RIGHT INFO.
%\lecture{**LECTURE-NUMBER**}{**DATE**}{**LECTURERS**}{**SCRIBE**}
\lecture{7}{Yoyo Tricks with AES}{Maria Francis and M. V. Panduranga Rao}{Gautam Singh}{16 April 2025}
%\footnotetext{These notes are partially based on those of Nigel Mansell.}

%All figures are to be placed in a separate folder named ``images''

% **** YOUR NOTES GO HERE:

\section{Introduction}

The yoyo game was introduced by Biham et al. in the cryptanalysis of SKIPJACK in
1998. It is based on making new pairs of plaintexts and ciphertexts that
preserve a certain property inherited from the original pair. This leads to a
partition of the plaintext and ciphertext spaces where each partition is closed
under exchange operations. 

The yoyo game is quite similar to the boomerang attack, and has been used to
build distinguishers for Feistel networks. They can also attack substitution
permutation networks (SPNs) that iterate a round function \(A \circ S\) where
\(A\) is an affine transformation and \(S\) is a non-linear S-box layer. 

\section{Yoyo Analysis of Generic SPNs}

For simplicity, we analyse permutations on \(\bF_q^n\) for \(q = 2^k\) of the
form \(F\brak{x} = S \circ L \circ S \circ L \circ S\), where \(L\) is a linear
transformation as opposed to an affine transformation. An element of \(\bF_q^n\)
is of the form \(\alpha = \brak{\alpha_0, \alpha_1, \ldots, \alpha_n}\) where
each \(\alpha_i \in \bF_q\) is called a \emph{word}.

To compare two differences according to their zero positions, we use the
following.

\begin{definition}[Zero Difference Pattern]
    \label{def:zero-diff}
    Let \(\alpha \in \bF_q^n\). Then, the zero difference pattern of \(\alpha\)
    is given by
    \begin{equation}
        \nu\brak{\alpha} \triangleq \brak{z_0, z_1, \ldots, z_{n - 1}}
        \label{eq:zero-diff}
    \end{equation}
    where \(z_i = 1\) if \(\alpha_i = 0\) or \(z_i = 0\) otherwise.
\end{definition}

Clearly, \(\nu\brak{\alpha} \in \bF_2^n\) and the complement of the
zero-difference pattern is called the \emph{activity pattern}. Linear
transformations do not preserve the zero difference pattern, but permutations
do.

\begin{lemma}
    \label{lem:zero-diff-perm}
    For two states \(\alpha, \beta \in \bF_q^n\), the zero pattern of their
    difference is preserved through \(S\). Mathematically,
    \begin{equation}
        \nu\brak{\alpha \oplus \beta} = \nu\brak{S\brak{\alpha} \oplus S\brak{\beta}}.
        \label{eq:zero-diff-perm}
    \end{equation}
\end{lemma}
\begin{proof}
    This is evident from the fact that \(\alpha_i \oplus \beta_i = 0 \iff
    s\brak{\alpha_i} \oplus s\brak{\beta_i} = 0\) since \(s\) is a permutation.
\end{proof}

We make extensive use of the following definition.

\begin{definition}
    \label{def:rho-v}
    For a vector \(v \in \bF_2^n\) and a pair of states \(\alpha, \beta \in
    \bF_q^n\) define \(\rho^v\brak{\alpha, \beta} \in \bF_q^n\) where
    \begin{equation}
        \rho^v\brak{\alpha, \beta}_i \triangleq \alpha_iv_i \oplus \beta_i\brak{v_i \oplus 1} =
        \begin{cases}
            \alpha_i & v_i = 1 \\
            \beta_i & v_i = 0
        \end{cases}.
        \label{eq:rho-v-def}
    \end{equation}
\end{definition}

From the definition it is evident that

\begin{equation}
    \rho^v\brak{\alpha, \beta} \oplus \rho^v\brak{\beta, \alpha} = \alpha \oplus \beta.
    \label{eq:rho-xor}
\end{equation}

The function \(\rho^v\) has some interesting properties which are stated and
proved below.

\begin{lemma}
    \label{lem:rho-perm}
    Let \(\alpha, \beta \in \bF_q^n\) and \(v \in \bF_2^n\). Then, \(\rho\)
    commutes with the S-box layer. Mathematically,
    \begin{equation}
        \rho^v\brak{S\brak{\alpha}, S\brak{\beta}} = S\brak{\rho^v\brak{\alpha, \beta}}
        \label{eq:rho-s-invariant}
    \end{equation}
    and thus
    \begin{equation}
        S\brak{\alpha} \oplus S\brak{\beta} = S\brak{\rho^v\brak{\alpha, \beta}} \oplus S\brak{\rho^v\brak{\beta, \alpha}}.
    \end{equation}
\end{lemma}
\begin{proof}
    \(S\) operates on each word independently and the result follows immediately
    from \autoref{def:rho-v}.
\end{proof}

\begin{lemma}
    \label{lem:rho-lin}
    For a linear transformation \(L\brak{x} = L\brak{x_0, x_1, \ldots, x_{n -
    1}}\) acting on \(n\) words we have
    \begin{equation}
        L\brak{\alpha} \oplus L\brak{\beta} = L\brak{\rho^v\brak{\alpha, \beta}} \oplus L\brak{\rho^v\brak{\beta, \alpha}}
    \end{equation}
    for any \(v \in \bF_2^n\).
\end{lemma}
\begin{proof}
    Using \eqref{eq:rho-xor} and the linearity of \(L\), we have
    \begin{equation}
        L\brak{\alpha} \oplus L\brak{\beta} = L\brak{\alpha \oplus \beta} = L\brak{\rho^v\brak{\alpha, \beta} \oplus \rho^v\brak{\beta, \alpha}} = L\brak{\rho^v\brak{\alpha, \beta}} \oplus L\brak{\rho^v\brak{\beta, \alpha}}
    \end{equation}
    as desired.
\end{proof}

Using \autoref{lem:rho-perm} and \autoref{lem:rho-lin}, we have

\begin{equation}
    L\brak{S\brak{\alpha}} \oplus L\brak{S\brak{\beta}} = L\brak{S\brak{\rho^v\brak{\alpha, \beta}}} \oplus L\brak{S\brak{\rho^v\brak{\beta, \alpha}}},
    \label{eq:l-comp-s}
\end{equation}

however switching \(S\) and \(L\) does not guarantee equality in
\eqref{eq:l-comp-s}.

Observe that the zero difference pattern does not change when we apply \(L\) or
\(S\) to any pair \(\alpha^\prime = \rho^v\brak{\alpha, \beta}\) and
\(\beta^\prime = \rho^v\brak{\beta, \alpha}\). Thus, unlike \eqref{eq:l-comp-s},
it can be shown that

\begin{equation}
    \nu\brak{S\brak{L\brak{\alpha}} \oplus S\brak{L\brak{\beta}}} = \nu\brak{S\brak{L\brak{\rho^v\brak{\alpha, \beta}}} \oplus S\brak{L\brak{\rho^v\brak{\beta, \alpha}}}}.
    \label{eq:nu-s-comp-l}
\end{equation}

In other words, although equality may not hold, the differences are zero in
exactly the same positions when \(S \circ L\) is applied.

The above results can be summarised as \autoref{thm:yoyo}, which is heavily used
in the yoyo attack.

\begin{theorem}
    \label{thm:yoyo}
    Let \(\alpha, \beta \in \bF_q^n\) and \(\alpha^\prime = \rho^v\brak{\alpha,
    \beta}, \beta^\prime = \rho^v\brak{\beta, \alpha}\). Then,
    \begin{equation}
        \nu\brak{S \circ L \circ S\brak{\alpha} \oplus S \circ L \circ S\brak{\beta}} = \nu\brak{S \circ L \circ S\brak{\alpha^\prime} \oplus S \circ L \circ S\brak{\beta^\prime}}.
        \label{eq:yoyo}
    \end{equation}
\end{theorem}
\begin{proof}
    The proof follows from the following observations.
    \begin{enumerate}
        \item \autoref{lem:rho-perm} gives \(S\brak{\alpha} \oplus S\brak{\beta}
        = S\brak{\alpha^\prime} \oplus S\brak{\beta^\prime}\).
        \item The linearity of \(L\) gives \(L\brak{S\brak{\alpha}} \oplus
        L\brak{S\brak{\beta}} = L\brak{S\brak{\alpha^\prime}} \oplus
        L\brak{S\brak{\beta^\prime}}\).
        \item Finally, \autoref{lem:zero-diff-perm} gives \eqref{eq:yoyo}.
    \end{enumerate} 
\end{proof}

\subsection{Yoyo Distinguisher for Two Generic SP-Rounds}

Two generic SP rounds can be represented as \(G_2^\prime = L \circ S \circ L
\circ S\), where the last linear layer can be removed to instead represent it as
\(G_2 = S \circ L \circ S\). If we fix a pair of plaintexts \(p^0, p^1\) with a
paritcular zero difference pattern \(\nu\brak{p^0 \oplus p^1}\), then from the
corresponding ciphertexts \(c^0, c^1\), we can construct another pair of new
ciphertexts \(c^{\prime 0}, c^{\prime 1}\) such that their decrypted plaintexts
\(p^{\prime 0}, p^{\prime 1}\) also have the same zero difference pattern. This
follows directly from \autoref{thm:yoyo} and holds with probability 1.

\begin{theorem}[Generic Yoyo Game for Two SP-Rounds]
    \label{thm:yoyo-2-rounds}
    Let \(p^0 \oplus p^1 \in \bF_q^n\), \(c^0 = G_2\brak{p^0}\) and \(c^1 =
    G_2\brak{p^1}\). Then for any \(v \in bF_2^n\), let \(c^{\prime 0} =
    \rho^v\brak{c^0, c^1}\) and \(c^{\prime 1} = \rho^v\brak{c^1, c^0}\). Then,
    \begin{equation}
        \nu\brak{G_2^{-1}\brak{c^{\prime 0}} \oplus G_2^{-1}\brak{c^{\prime 1}}} = \nu\brak{p^{\prime 0} \oplus p^{\prime 1}} = \nu\brak{p^0 \oplus p^1}.
        \label{eq:yoyo-2-rounds}
    \end{equation}
\end{theorem}
\begin{proof}
    Since \(S^{-1}\) is also a permutation and \(L^{-1}\) is a linear
    transformation, we invoke \autoref{thm:yoyo} on \(G_2^{-1} = S^{-1} \circ
    L^{-1} \circ S^{-1}\) to obtain \eqref{eq:yoyo-2-rounds}.
\end{proof}

\autoref{thm:yoyo-2-rounds} gives us a straightforward distinguisher for two
generic SP-rounds requiring two plaintexts and two adaptively chosen
ciphertexts. A random permutation would not give back a pair of decrypted
plaintexts that still have the same zero difference pattern with very high
probability. One can go the other way to generate two ciphertexts and then
observe the ciphertexts of the adaptively chosen plaintexts.

\subsection{Analysis of Three Generic SP-Rounds}



\end{document}
