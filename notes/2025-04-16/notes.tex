% To familiarize yourself with this template, the body contains
% some examples of its use.  Look them over.  Then you can
% run LaTeX on this file.  After you have LaTeXed this file then
% you can look over the result either by printing it out with
% dvips or using xdvi.
%

\documentclass[twoside]{article}
%\usepackage{soul}
\usepackage{./lecnotes_macros}


\begin{document}
%FILL IN THE RIGHT INFO.
%\lecture{**LECTURE-NUMBER**}{**DATE**}{**LECTURERS**}{**SCRIBE**}
\lecture{7}{Yoyo Tricks with AES}{Maria Francis and M. V. Panduranga Rao}{Gautam Singh}{16 April 2025}
%\footnotetext{These notes are partially based on those of Nigel Mansell.}

%All figures are to be placed in a separate folder named ``images''

% **** YOUR NOTES GO HERE:

\section{Introduction}

The yoyo game was introduced by Biham et al. in the cryptanalysis of SKIPJACK in
1998. It is based on making new pairs of plaintexts and ciphertexts that
preserve a certain property inherited from the original pair. This leads to a
partition of the plaintext and ciphertext spaces where each partition is closed
under exchange operations. 

The yoyo game is quite similar to the boomerang attack, and has been used to
build distinguishers for Feistel networks. They can also attack substitution
permutation networks (SPNs) that iterate a round function \(A \circ S\) where
\(A\) is an affine transformation and \(S\) is a non-linear S-box layer. 

\section{Yoyo Analysis of Generic SPNs}

For simplicity, we analyse permutations on \(\bF_q^n\) for \(q = 2^k\) of the
form \(F\brak{x} = S \circ L \circ S \circ L \circ S\), where \(L\) is a linear
transformation as opposed to an affine transformation. An element of \(\bF_q^n\)
is of the form \(\alpha = \brak{\alpha_0, \alpha_1, \ldots, \alpha_n}\) where
each \(\alpha_i \in \bF_q\) is called a \emph{word}.

To compare two differences according to their zero positions, we use the
following.

\begin{definition}[Zero Difference Pattern]
    \label{def:zero-diff}
    Let \(\alpha \in \bF_q^n\). Then, the zero difference pattern of \(\alpha\)
    is given by
    \begin{equation}
        \nu\brak{\alpha} \triangleq \brak{z_0, z_1, \ldots, z_{n - 1}}
        \label{eq:zero-diff}
    \end{equation}
    where \(z_i = 1\) if \(\alpha_i = 0\) or \(z_i = 0\) otherwise.
\end{definition}

Clearly, \(\nu\brak{\alpha} \in \bF_2^n\) and the complement of the
zero-difference pattern is called the \emph{activity pattern}. Linear
transformations do not preserve the zero difference pattern, but permutations
do.

\begin{lemma}
    \label{lem:zero-diff-perm}
    For two states \(\alpha, \beta \in \bF_q^n\), the zero pattern of their
    difference is preserved through \(S\). Mathematically,
    \begin{equation}
        \nu\brak{\alpha \oplus \beta} = \nu\brak{S\brak{\alpha} \oplus S\brak{\beta}}.
        \label{eq:zero-diff-perm}
    \end{equation}
\end{lemma}
\begin{proof}
    This is evident from the fact that \(\alpha_i \oplus \beta_i = 0 \iff
    s\brak{\alpha_i} \oplus s\brak{\beta_i} = 0\) since \(s\) is a permutation.
\end{proof}

We make extensive use of the following definition.

\begin{definition}
    \label{def:rho-v}
    For a vector \(v \in \bF_2^n\) and a pair of states \(\alpha, \beta \in
    \bF_q^n\) define \(\rho^v\brak{\alpha, \beta} \in \bF_q^n\) where
    \begin{equation}
        \rho^v\brak{\alpha, \beta}_i \triangleq \alpha_iv_i \oplus \beta_i\brak{v_i \oplus 1} =
        \begin{cases}
            \alpha_i & v_i = 1 \\
            \beta_i & v_i = 0
        \end{cases}.
        \label{eq:rho-v-def}
    \end{equation}
\end{definition}

From the definition it is evident that
\begin{equation}
    \rho^v\brak{\alpha, \beta} \oplus \rho^v\brak{\beta, \alpha} = \alpha \oplus \beta.
    \label{eq:rho-xor}
\end{equation}
The function \(\rho^v\) has some interesting properties which are stated and
proved below.

\begin{lemma}
    \label{lem:rho-perm}
    Let \(\alpha, \beta \in \bF_q^n\) and \(v \in \bF_2^n\). Then, \(\rho\)
    commutes with the S-box layer. Mathematically,
    \begin{equation}
        \rho^v\brak{S\brak{\alpha}, S\brak{\beta}} = S\brak{\rho^v\brak{\alpha, \beta}}
        \label{eq:rho-s-invariant}
    \end{equation}
    and thus
    \begin{equation}
        S\brak{\alpha} \oplus S\brak{\beta} = S\brak{\rho^v\brak{\alpha, \beta}} \oplus S\brak{\rho^v\brak{\beta, \alpha}}.
    \end{equation}
\end{lemma}
\begin{proof}
    \(S\) operates on each word independently and the result follows immediately
    from \autoref{def:rho-v}.
\end{proof}

\begin{lemma}
    \label{lem:rho-lin}
    For a linear transformation \(L\brak{x} = L\brak{x_0, x_1, \ldots, x_{n -
    1}}\) acting on \(n\) words we have
    \begin{equation}
        L\brak{\alpha} \oplus L\brak{\beta} = L\brak{\rho^v\brak{\alpha, \beta}} \oplus L\brak{\rho^v\brak{\beta, \alpha}}
    \end{equation}
    for any \(v \in \bF_2^n\).
\end{lemma}
\begin{proof}
    Using \eqref{eq:rho-xor} and the linearity of \(L\), we have
    \begin{equation}
        L\brak{\alpha} \oplus L\brak{\beta} = L\brak{\alpha \oplus \beta} = L\brak{\rho^v\brak{\alpha, \beta} \oplus \rho^v\brak{\beta, \alpha}} = L\brak{\rho^v\brak{\alpha, \beta}} \oplus L\brak{\rho^v\brak{\beta, \alpha}}
    \end{equation}
    as desired.
\end{proof}

Using \autoref{lem:rho-perm} and \autoref{lem:rho-lin}, we have
\begin{equation}
    L\brak{S\brak{\alpha}} \oplus L\brak{S\brak{\beta}} = L\brak{S\brak{\rho^v\brak{\alpha, \beta}}} \oplus L\brak{S\brak{\rho^v\brak{\beta, \alpha}}},
    \label{eq:l-comp-s}
\end{equation}
however switching \(S\) and \(L\) does not guarantee equality in
\eqref{eq:l-comp-s}.

Observe that the zero difference pattern does not change when we apply \(L\) or
\(S\) to any pair \(\alpha^\prime = \rho^v\brak{\alpha, \beta}\) and
\(\beta^\prime = \rho^v\brak{\beta, \alpha}\). Thus, unlike \eqref{eq:l-comp-s},
it can be shown that
\begin{equation}
    \nu\brak{S\brak{L\brak{\alpha}} \oplus S\brak{L\brak{\beta}}} = \nu\brak{S\brak{L\brak{\rho^v\brak{\alpha, \beta}}} \oplus S\brak{L\brak{\rho^v\brak{\beta, \alpha}}}}.
    \label{eq:nu-s-comp-l}
\end{equation}
In other words, although equality may not hold, the differences are zero in
exactly the same positions when \(S \circ L\) is applied.

The above results can be summarised as \autoref{thm:yoyo}, which is heavily used
in the yoyo attack.

\begin{theorem}
    \label{thm:yoyo}
    Let \(\alpha, \beta \in \bF_q^n\) and \(\alpha^\prime = \rho^v\brak{\alpha,
    \beta}, \beta^\prime = \rho^v\brak{\beta, \alpha}\). Then,
    \begin{equation}
        \nu\brak{S \circ L \circ S\brak{\alpha} \oplus S \circ L \circ S\brak{\beta}} = \nu\brak{S \circ L \circ S\brak{\alpha^\prime} \oplus S \circ L \circ S\brak{\beta^\prime}}.
        \label{eq:yoyo}
    \end{equation}
\end{theorem}
\begin{proof}
    The proof follows from the following observations.
    \begin{enumerate}
        \item \autoref{lem:rho-perm} gives \(S\brak{\alpha} \oplus S\brak{\beta}
        = S\brak{\alpha^\prime} \oplus S\brak{\beta^\prime}\).
        \item The linearity of \(L\) gives \(L\brak{S\brak{\alpha}} \oplus
        L\brak{S\brak{\beta}} = L\brak{S\brak{\alpha^\prime}} \oplus
        L\brak{S\brak{\beta^\prime}}\).
        \item Finally, \autoref{lem:zero-diff-perm} gives \eqref{eq:yoyo}.
    \end{enumerate} 
\end{proof}

\subsection{Yoyo Distinguisher for Two Generic SP-Rounds}

Two generic SP rounds can be represented as \(G_2^\prime = L \circ S \circ L
\circ S\), where the last linear layer can be removed to instead represent it as
\(G_2 = S \circ L \circ S\). If we fix a pair of plaintexts \(p^0, p^1\) with a
paritcular zero difference pattern \(\nu\brak{p^0 \oplus p^1}\), then from the
corresponding ciphertexts \(c^0, c^1\), we can construct another pair of new
ciphertexts \(c^{\prime 0}, c^{\prime 1}\) such that their decrypted plaintexts
\(p^{\prime 0}, p^{\prime 1}\) also have the same zero difference pattern. This
follows directly from \autoref{thm:yoyo} and holds with probability 1.

\begin{theorem}[Generic Yoyo Game for Two SP-Rounds]
    \label{thm:yoyo-2-rounds}
    Let \(p^0 \oplus p^1 \in \bF_q^n\), \(c^0 = G_2\brak{p^0}\) and \(c^1 =
    G_2\brak{p^1}\). Then for any \(v \in bF_2^n\), let \(c^{\prime 0} =
    \rho^v\brak{c^0, c^1}\) and \(c^{\prime 1} = \rho^v\brak{c^1, c^0}\). Then,
    \begin{equation}
        \nu\brak{G_2^{-1}\brak{c^{\prime 0}} \oplus G_2^{-1}\brak{c^{\prime 1}}} = \nu\brak{p^{\prime 0} \oplus p^{\prime 1}} = \nu\brak{p^0 \oplus p^1}.
        \label{eq:yoyo-2-rounds}
    \end{equation}
\end{theorem}
\begin{proof}
    Since \(S^{-1}\) is also a permutation and \(L^{-1}\) is a linear
    transformation, we invoke \autoref{thm:yoyo} on \(G_2^{-1} = S^{-1} \circ
    L^{-1} \circ S^{-1}\) to obtain \eqref{eq:yoyo-2-rounds}.
\end{proof}

\autoref{thm:yoyo-2-rounds} gives us a straightforward distinguisher for two
generic SP-rounds requiring two plaintexts and two adaptively chosen
ciphertexts. A random permutation would not give back a pair of decrypted
plaintexts that still have the same zero difference pattern with very high
probability. One can go the other way to generate two ciphertexts and then
observe the ciphertexts of the adaptively chosen plaintexts.

\subsection{Analysis of Three Generic SP-Rounds}

As before, three SP rounds can be modeled as \(G_3 = S \circ L \circ S \circ L
\circ S\). For two states \(\alpha\) and \(\beta\), using
\autoref{thm:yoyo-2-rounds}, it follows that
\begin{equation}
    \nu\brak{G_2^{-1}\brak{\rho^v\brak{G_2\brak{\alpha}, G_2\brak{\beta}}} \oplus G_2^{-1}\brak{\rho^v\brak{G_2\brak{\beta}, G_2\brak{\alpha}}}} = \nu\brak{\alpha \oplus \beta}.
\end{equation}
Since \(G_2\) and \(G_2^{-1}\) have identical forms, we have
\begin{equation}
    \nu\brak{G_2\brak{\rho^v\brak{G_2^{-1}\brak{\alpha}, G_2^{-1}\brak{\beta}}} \oplus G_2\brak{\rho^v\brak{G_2^{-1}\brak{\beta}, G_2^{-1}\brak{\alpha}}}} = \nu\brak{\alpha \oplus \beta}.
\end{equation}
Finally, from \autoref{lem:rho-perm}, zero difference patterns are preserved
through an S-box layer. Putting it all together gives us the following theorem.

\begin{theorem}[Generic Yoyo Game for 3 SP-Rounds]
    \label{thm:yoyo-3-rounds}
    Let \(G_3 = S \circ L \circ S \circ L \circ S\). If \(p^0, p^1 \in \bF_q^n\)
    and \(c^0 = G_3\brak{p^0}\), \(c^1 = G_3\brak{p^1}\), then
    \begin{equation}
        \nu\brak{G_2\brak{\rho^{v_1}\brak{p^0, p^1}} \oplus G_2\brak{\rho^{v_1}\brak{p^1, p^0}}} = \nu\brak{G_2^{-1}\brak{\rho^{v_2}\brak{c^0, c^1}} \oplus G_2^{-1}\brak{\rho^{v_2}\brak{c^1, c^0}}}
    \end{equation}
    for any \(v_1, v_2 \in \bF_2^n\). Moreover, for any \(z \in \bF_2^n\),
    define \(R_P\brak{z} \triangleq \cbrak{\brak{p^0, p^1} \mid
    \nu\brak{G_2\brak{p^0} \oplus G_2\brak{p^1}} = z}\) and \(R_C\brak{z}
    \triangleq \cbrak{\brak{c^0, c^1} \mid \nu\brak{G_2^{-1}\brak{c^0} \oplus
    G_2^{-1}\brak{c^1}} = z}\). Then,
    \begin{equation}
        \brak{G_3\brak{\rho^v\brak{p^0, p^1}}, G_3\brak{\rho^v\brak{p^1, p^0}}} \in R_C\brak{z}
    \end{equation}
    for any \(\brak{p^0, p^1} \in R_P\brak{z}\), while
    \begin{equation}
        \brak{G_3^{-1}\brak{\rho^v\brak{c^0, c^1}}, G_3^{-1}\brak{\rho^v\brak{c^1, c^0}}} \in R_C\brak{z}
    \end{equation}
    for any \(\brak{c^0, c^1} \in R_C\brak{z}\).
\end{theorem}

Thus, given a pair in \(R_P\brak{z}\), we can generate new pairs that belong to
\(R_P\brak{z}\) and \(R_C\brak{z}\) with probability 1. 

The key idea behind a distinguisher for three SP-rounds is to get a pair with a
particular Hamming weight of the zero difference pattern and then detect this
occurence. The probability that a random pair of plaintexts has a sum with
nonzero difference pattern containing exactly \(m\) zeros is
\(\binom{n}{m}\frac{\brak{q-1}^m}{q^n}\) where \(q = 2^k\). Thus, we need to
test approximately the inverse of that number of pairs to find one correct pair.

Detecting a correct pair is more involved. Suppose \(\brak{p_1, p_2} \in
R_P\brak{z}\) and let the respective ciphertexts be \(\brak{c_1, c_2}\). Let
\(A\) be the affine layer in an SASAS construction. Assume that
\(S^{-1}\brak{c^0} = x \oplus z\) and \(S^{-1}\brak{c^1} = y \oplus z\), where
\(A^{-1}\brak{x}, A^{-1}\brak{y}\) and \(A^{-1}\brak{z}\) are non-zero only in
the positions where \(z\) is zero. It follows that \(x\) and \(y\) belong to a
linear subspace \(U\) of dimension \(n - m\) while \(z\) belongs to the
complementary linear subspace \(V\) of dimension \(m\) such that \(U \oplus V =
\bF_q^n\). Thus, we need to investigate whether \(c_1 \oplus c_2 = S\brak{x
\oplus z} \oplus S\brak{y \oplus z}\) has some distinguishing properties.

\section{Applications to AES}

\subsection{Preliminaries}

The round function in AES is represented as operations over \(\bF_q^{4 \times
4}\) where \(q = 2^8\). One round of AES can be written as \(R = AK \circ MC
\circ SR \circ SB\). Since we are working with differences, we can strip \(AK\)
operations. Further, \(SR\) and \(SB\) commute. Thus, two rounds of AES can be
written as
\begin{equation}
    R^{2\prime} = MC \circ SR \circ \brak{SB \circ MC \circ SB} \circ SR
    \label{eq:aes-2-rounds}
\end{equation}
where \(S = SB \circ MC \circ SB\) can be thought of as four parallel 32-bit
super S-boxes. Finally, the initial \(SR\) has no effect, thus we can rewrite
two rounds of AES as
\begin{equation}
    R^2 = MC \circ SR \circ S.
\end{equation}
Now, considering \(S = SB \circ MC \circ SB\) and \(L = SR \circ MC \circ SR\),
four rounds of AES can be represented using \eqref{eq:aes-2-rounds} as
\begin{equation}
    R^{4\prime} = MC \circ SR \circ S \circ L \circ S \circ SR
\end{equation}
which ends up becoming
\begin{equation}
    R^4 = S \circ L \circ S.
\end{equation}
This can also be used to show that a lower bound on the number of active S boxes
over four rounds is 25. This is because the number of active super S-boxes is 5
due to the linear layer and there are at least 5 active S boxes insided a super
S-box due to the MixColumns matrix.

Similarly, six rounds of AES can be written as
\begin{equation}
    R^6 = S \circ L \circ S \circ L \circ S.
\end{equation}
For convenience, we introduce the following definition.

\begin{definition}
    \label{def:q}
    Let \(Q \triangleq SB \circ MC \circ SR\) and \(Q^\prime \triangleq SR \circ
    MC \circ SB\).
\end{definition}

Since two rounds of AES correspond to one generic SPN round, we must exploit the
properties of one AES round to create distinguishers for an odd number of
rounds. Adding another round at the end of \eqref{eq:aes-2-rounds}, three rounds
of AES can be written as \(Q \circ S\) and similarly, five rounds of AES can be
written as \(S \circ L \circ S \circ Q^\prime\).

We now consider some properties of \(Q\) and \(Q^\prime\). For a binary vector
\(z \in \bF_4^2\) of weight \(t\), let \(V_z\) denote the subspace of
\(q^{4\cdot\brak{4 - t}}\) states \(x = \brak{x_0, x_1, x_2, x_3}\) where \(x_i
\in \bF_q^4\) if \(z_i = 0\) or \(x_i = 0\) otherwise. For any state \(a =
\brak{a_0, a_1, a_2, a_3}\), let
\begin{equation}
    T_{z, a} \triangleq \cbrak{Q\brak{a \oplus x} \mid x \in V_z}.
    \label{eq:tza-def}
\end{equation}
Note that the sets \(T_{z, a}\) depend on keyed functions such as those depicted
in \autoref{def:q}. Let \(H_i\) denote the image of the \(i\)-th word in
\(SR\brak{a \oplus x}\) for \(x \in V_z\). Notice that \(\abs{H_i} = q^{4 -
t}\). Define
\begin{equation}
    T_i^{z, a} \triangleq SB \circ MC\brak{H_i}.
    \label{eq:tiza-def}
\end{equation}
Since \(SB\) and \(MC\) operate on each word individually, we obtain the
following.

\begin{lemma}
    \label{lem:tz-rel}
    The set \(T_{z, a}\) satisfies
    \begin{equation}
        T_{z, a} = T_0^{z, a} \times T_1^{z, a} \times T_2^{z, a} \times T_3^{z, a}
        \label{eq:tz-rel}
    \end{equation}
    where \(\abs{T_i^{z, a}} = q^{4 - hw\brak{z}}\), with \(hw\brak{z}\)
    denoting the Hamming weight of \(z\).
\end{lemma}
\begin{proof}
    Notice that each word of \(Q\brak{a \oplus x}\) contributes one byte to each
    word after \(SR\). Thus, if \(4 - t\) words are nonzero, it follows that
    each word after \(SR\) can take exactly \(q^{4 - t}\) values. Thus,
    \(T_i^{z, a} = SB \circ MC\brak{H_i}\).
\end{proof}

A similar property can be derived for \(Q^\prime\) and its inverse as well.

Later algorithms make use of the primitive \textsc{SimpleSWAP} algorithm shown
in \autoref{alg:simple-swap} to perform the yoyo itself. Further, note that we
consider \(r\)-round AES encryption wihtout first \(SR\) and last \(SR \circ
MC\).

\begin{algorithm}
    \caption{Swaps the first word where texts are different and returns one word.}
    \label{alg:simple-swap}
    \begin{algorithmic}[1]
        \Function{SimpleSWAP}{\(x^0\), \(x^1\)} \Comment{\(x^0 \ne x^1\)}
            \State \(x^{\prime 0} \gets x^{\prime 1}\)
            \For{\(i\) from 0 to 3}
                \If{\(x_i^0 \ne x_i^1\)}
                    \State \(x_i^{\prime 0} \gets x_i^{\prime 1}\)
                    \State \Return \(x^{\prime 0}\)
                \EndIf
            \EndFor
        \EndFunction
    \end{algorithmic}
\end{algorithm}

\subsection{Yoyo Distinguisher for Three Rounds of AES}

We have seen that three rounds of AES can be written as \(R^3 = Q \circ S\). We
use \autoref{lem:tz-rel} to create the distinguisher. Consider plaintexts \(p^0,
p^1\) such that \(z = \nu\brak{p^0 \oplus p^1}\) and \(t = hw\brak{z}\). Using
\autoref{lem:zero-diff-perm}, we see that \(\nu\brak{S\brak{p^0} \oplus
S\brak{p^1}} = \nu\brak{p^0 \oplus p^1}\). Then, from \autoref{lem:tz-rel},
\(Q\brak{S\brak{p^0}} = c^0\) and \(Q\brak{S\brak{p^1}} = c^1\) also belong to
\(T_{z, a}\). Further, each word is drawn from the subsets \(T_i^{z, a}\). In
paritcular, we have
\begin{equation}
    T_{z, a}^\prime = \cbrak{c_0^0, c_0^1} \times \cbrak{c_1^0, c_1^1} \times \cbrak{c_2^0, c_2^1} \times \times \cbrak{c_3^0, c_3^1} \subset T_{z, a}.
\end{equation}
where the size of \(T_{z, a}^\prime\) is at most \(2^4\) and \(\cbrak{c_i^0,
c_i^1} \subset T_i^{z, a}\). Thus, any other ciphertext \(c^\prime \ne c^0,
c^1\) from \(T_{z, a}^\prime\) satisfies \(\nu\brak{Q^{-1}\brak{c^\prime} \oplus
S\brak{p^0}} = \nu\brak{Q^{-1}\brak{c^\prime} \oplus S\brak{p^1}} =
\nu\brak{S\brak{p^0} \oplus S\brak{p^1}}\). In particular, we have
\(\nu\brak{R^{-3}\brak{c^\prime} \oplus p^0} = \nu\brak{R^{-3}\brak{c^\prime}
\oplus p^1} = \nu\brak{p^0 \oplus p^1}\). If a random permutation were used, the
chosen ciphertext \(c^\prime\) would satisfy this condition with probability
\(2^{-96}\). The distinguisher is summarised in \autoref{alg:aes-3-rounds}.
Clearly, only two plaintexts and one adaptively chosen ciphertext is required.

\begin{algorithm}
    \caption{Distinguisher for Three Rounds of AES}
    \label{alg:aes-3-rounds}
    \begin{algorithmic}[1]
        \Require{Plaintexts \(p^0, p^1\) with \(hw\brak{\nu\brak{p^0 \oplus
        p^1}} = 3\)}
        \Ensure{1 for AES, -1 otherwise}
        \State \(c^0 \gets enc_k\brak{p^0, 3}\), \(c^1 \gets enc_k\brak{p^1,
        3}\)
        \State \(c^\prime \gets \textsc{SimpleSWAP}\brak{c^0, c^1}\)
        \State \(p^\prime \gets dec_k\brak{c^\prime, 3}\)
        \If{\(\nu\brak{p^0 \oplus p^1} = \nu\brak{p^\prime \oplus p^1}\)}
            \State \Return 1
        \Else
            \State \Return -1
        \EndIf
    \end{algorithmic}
\end{algorithm}

\subsection{Yoyo Distinguisher for Four Rounds of AES}

Four rounds of AES can be represented as \(R^4 = S \circ L \circ S\) after
simplification. We make use of \autoref{thm:yoyo-2-rounds} to create the
distinguisher. Again, the new ciphertexts are created by simply exchanging words
between the two obtined ciphertexts, as shown in \autoref{alg:aes-4-rounds}.
This distinguisher requires two plaintexts and two adaptively chosen
ciphertexts.

\begin{algorithm}
    \caption{Distinguisher for Four Rounds of AES}
    \label{alg:aes-4-rounds}
    \begin{algorithmic}[1]
        \Require{Plaintexts \(p^0, p^1\) with \(hw\brak{\nu\brak{p^0 \oplus
        p^1}} = 3\)}
        \Ensure{1 for AES, -1 otherwise}
        \State \(c^0 \gets enc_k\brak{p^0, 4}\), \(c^1 \gets enc_k\brak{p^1,
        4}\)
        \State \(c^{\prime 0} \gets \textsc{SimpleSWAP}\brak{c^0, c^1}\),
        \(c^{\prime 1} \gets \textsc{SimpleSWAP}\brak{c^1, c^0}\)
        \State \(p^{\prime 0} \gets dec_k\brak{c^{\prime 0}, 4}\), \(p^{\prime
        1} \gets dec_k\brak{c^{\prime 1}, 4}\)
        \If{\(\nu\brak{p^0 \oplus p^1} = \nu\brak{p^{\prime 0} \oplus p^{\prime
        1}}\)}
            \State \Return 1
        \Else
            \State \Return -1
        \EndIf
    \end{algorithmic}
\end{algorithm}

\subsection{Yoyo Distinguisher for Five Rounds of AES}

Five rounds of AES can be written as \(R^5 = S \circ L \circ S \circ Q^\prime =
R^4 \circ Q^\prime\). If the difference between two plaintexts after
\(Q^\prime\) is zero in \(t\) words, we can apply the yoyo game and get new
plaintext pairs that are zero in exactly the same words after \(Q^\prime\) and
thus, reside in the same sets by \autoref{lem:tz-rel}. In paritcular, if a pair
of plaintexts \(p^0, p^1\) are encrypted through \(Q^\prime\) to a pair of
intermediate states with zero difference in 3 out of 4 words, then they have
probability \(q^{-1}\) of having the same value in a particular word, since
\(\abs{T_i^{z, a}} = q^{4 - 3} = q\) by \autoref{lem:tz-rel}. Another property,
this time of the MixColumns matrix can be exploited to get a tighter bound,
which is stated as \autoref{lem:mc-active}.

\begin{lemma}
    \label{lem:mc-active}
    Let \(M\) denote a \(4 \times 4\) MixColumns matrix and \(x \in \bF_q^4\).
    If \(t\) bytes in \(x\) are zero, then \(x \cdot M\) or \(x \cdot M^{-1}\)
    cannot contain \(4 - t\) or more zeros.
\end{lemma}

This is used to prove \autoref{thm:q-weight}.

\begin{theorem}
    \label{thm:q-weight}
    Let \(a\) and \(b\) denote two states where \(\nu\brak{Q^\prime\brak{a}
    \oplus Q^\prime\brak{b}}\) has weight \(t\). Then, the probability that any
    \(4 - t\) bytes are simultaneously zero in a word in the difference \(a
    \oplus b\) is \(q^{t - 4}\). When this happens, all bytes in the difference
    are zero.
\end{theorem}
\begin{proof}
    From \autoref{lem:tz-rel}, words in the same position of \(a\) and \(b\) are
    drawn from \(T_i^{z, a}\) with size \(q^{4 - t}\). Thus, words in the same
    position are equal with probability \(q^{t - 4}\). Since \(t\) words are
    zero in \(Q^\prime\brak{a} \oplus Q^\prime\brak{b}\), \(t\) bytes are zero
    in each word of \(SR^{-1}\brak{Q^\prime\brak{a}} \oplus
    SR^{-1}\brak{Q^\prime\brak{b}}\). From \autoref{lem:mc-active}, we see that
    \(4 - t\) bytes cannot be zero in each word after \(MC^{-1}\). This is
    preserved through \(SB^{-1}\) and XOR with the round key.
\end{proof}

To build the distinguisher, we need to create enough plaintext pairs so that
there will be exactly \(t\) zeros after the application of \(Q^\prime\). Notice
that two equal columns remain equal on applying \(MC \circ SB\). Thus, the
adversary chooses pairs \(\brak{p^0, p^1}\) which are nonzero in exactly one
word and tries enough pairs until the corresponding active word after applying
\(MC \circ SB\) on that word has \(t\) zero bytes. This would imply
\(Q^\prime\brak{p^0} \oplus Q^\prime\brak{p^1}\) has \(t\) zero words. Playing
the yoyo game on \(R^4\) will return at most 7 new plaintext pairs which have
the same zero difference pattern after one round and obey
\autoref{thm:q-weight}. This is used to distinguish five-round AES from a random
permutation.

The probability that a pair \(\brak{p^0, p^1}\) with a zero difference pattern
of weight 3 has a zero difference pattern of weight \(t\) when encrypted through
\(Q^\prime\) is given by
\begin{equation}
    p_b\brak{t} = \binom{4}{t}q^{-t}
\end{equation}
where \(q = 2^8\). Thus, we require \(p_b\brak{t}^{-1}\) pairs to get one such
pair. To distinguish a correct pair, notice that for a random pair of
plaintexts, the probability that \(4 - t\) bytes are zero simultaneously in any
of the 4 words is approximately
\begin{equation}
    4p_b\brak{4 - t} = 4 \cdot \binom{4}{t} \cdot q^{t - 4}
\end{equation}
while for a correct pair it is \(4 \cdot q^{t - 4}\). Thus, each pair of
plaintexts requires \(\frac{p_b\brak{4 - t}^{-1}}{4}\) plaintext pairs using the
yoyo game. Thus, the total data complexity is
\begin{equation}
    2 \cdot \brak{p_b\brak{t}^{-1} \cdot \brak{4 \cdot p_b\brak{4 - t}}^{-1}} = \frac{p_b\brak{t} \cdot p_b\brak{4 - t}^{-1}}{2}.
\end{equation}
For \(t = 2\), the data complexity is minimum at approximately \(2^{25.8}\). The
overall distinguisher is shown in \autoref{alg:aes-5-rounds}.

\begin{algorithm}
    \caption{Distinguisher for Five Rounds of AES}
    \label{alg:aes-5-rounds}
    \begin{algorithmic}[1]
        \Ensure{1 for AES, -1 otherwise}
        \State \(cnt1 \gets 0\).
        \While {\(cnt1 < 2^{13.4}\)}
            \State \(cnt1 \gets cnt1 + 1\).
            \State \(p^0, p^1 \gets\) generate random pair with
            \(hw\brak{\nu\brak{p^0 \oplus p^1}} = 3\).
            \State \(cnt2 \gets 0\), \(WrongPair \gets False\).
            \While {\(cnt2 < 2^{11.4}\ \&\ WrongPair = False\)}
                \State \(cnt2 \gets cnt2 + 1\).
                \State \(c^0 \gets enc_k\brak{p^0, 5}\), \(c^1 \gets
                enc_k\brak{p^1, 5}\).
                \State \(c^{\prime 0} \gets \textsc{SimpleSWAP}\brak{c^0,
                c^1}\), \(c^{\prime 1} \gets \textsc{SimpleSWAP}\brak{c^1,
                c^0}\).
                \State \(p^{\prime 0} \gets dec_k\brak{c^{\prime 0}, 5}\),
                \(p^{\prime 1} \gets dec_k\brak{c^{\prime 1}, 5}\).
                \For{\(i\) from 0 to 3}
                    \If{\(hw\brak{\nu\brak{p_i}} \ge 2\)}
                        \State \(WrongPair = True\)
                    \EndIf
                \EndFor
                \State \(p^{\prime 0} \gets \textsc{SimpleSWAP}\brak{p^0,
                p^1}\), \(p^{\prime 1} \gets \textsc{SimpleSWAP}\brak{p^1,
                p^0}\).
            \EndWhile
            \If{\(WrongPair = False\)}
                \State \Return 1 \Comment{Did not find difference with two or
                more zeros.}
            \EndIf
        \EndWhile
        \State \Return -1
    \end{algorithmic}
\end{algorithm}

\subsection{A Five Round Key Recovery Yoyo on AES}

The aim of this attack is to find the first round key \(k_0\) XORed in front
of \(R^5\). The MixColumns matrix \(M\) in AES is given by
\begin{equation}
    M = \myvec{
        \alpha & \alpha \oplus 1 & 1 & 1 \\
        1 & \alpha & \alpha \oplus 1 & 1 \\
        1 & 1 & \alpha & \alpha \oplus 1 \\
        \alpha \oplus 1 & 1 & 1 & \alpha
    }.
\end{equation}

The function \(MC \circ SB\) works on each word independently, thus assume we
pick two plaintexts \(p^0\) and \(p^1\) where the first words are given by
\(p_0^0 = \brak{0, i, 0, 0}\) and \(p_0^1 = \brak{z, z \oplus i, 0, 0}\) where
\(z \in \bF_q \setminus \cbrak{0}\) and the three other words are equal. Let
\(k_0 = \brak{k_{0, 0}, k_{0, 1}, k_{0, 2}, k_{0, 3}}\) denote key bytes XORed
with the first word of the plaintext. The difference between the first words
after partial encryption of the two plaintexts \(MC \circ SB \circ AK\) becomes
\begin{align}
    \alpha b_0 \oplus \brak{\alpha \oplus 1}b_1 &= y_0 \\
    b_0 \oplus \alpha b_1 &= y_1 \\
    b_0 \oplus b_1 &= y_2 \label{eq:y2} \\
    \brak{\alpha \oplus 1}b_0 \oplus b_1 &= y_3
\end{align}
where \(b_0 = s\brak{k_{0, 0}} \oplus s\brak{z \oplus k_{0, 0}}\) and \(b_1 =
s\brak{k_{0, 1} \oplus i} \oplus s\brak{k_{0, 1} \oplus z \oplus i}\). In
paritcular, \eqref{eq:y2} can be written as
\begin{equation}
    s\brak{k_{0, 0}} \oplus s\brak{k_{0, 0} \oplus z} \oplus s\brak{k_{0, 1} \oplus i} \oplus s\brak{k_{0, 1} \oplus z \oplus i} = y_2.
\end{equation}

As \(i\) runs over all \(\bF_q\), we see that \(y_2 = 0\) for \(i \in
\cbrak{k_{0, 0} \oplus k_{0, 1}, k_{0, 0} \oplus k_{0, 1} \oplus z}\). Hence,
there will be at least two values of \(i\) for which \(y_2 = 0\). Define \(B = M
\circ s^4\) to be the action of \(MC \circ SB\) on one column, where \(s^4\) is
the concatenation of four S-boxes in parallel. We prepare a set \(\cP\) of
plaintexts \(p^0\) and \(p^1\) where \(p_0^0 = \brak{0, i, 0, 0}\) and \(p_0^1 =
\brak{z, z \oplus i, 0, 0}\). Let \(c^0, c^1\) be the respective ciphertexts.
Pick 5 new ciphertext pairs \(\brak{c^{\prime 0}, c^{\prime 1}} =
\brak{\rho^v\brak{c^0, c^1}, \rho^v\brak{c^1, c^0}}\) and let \(p^{\prime 0},
p^{\prime 1}\) be the respective plaintexts. A correct pair will satisfy
\begin{equation}
    B\brak{p_0^{\prime 0} \oplus k_0} \oplus B\brak{p_0^{\prime 1} \oplus k_0} = \brak{z_0, z_1, 0, z_3}.
    \label{eq:aes-5-rounds-check}
\end{equation}

The adversary can now test the remaining \(2^{24}\) candidate keys and find
whether the third byte of the first word is zero for all 5 pairs of plaintexts,
where \(k_{0, 0} \oplus k_{0, 1} \in \cbrak{i, i \oplus z}\) for known \(i\) and
\(z\). This holds for all 5 pairs at random with probability \(2^{-8 \cdot 5} =
2^{-40}\). Hence, a false positive might occur with probability \(2^{-16}\) when
testing \(2^{24}\) keys. This probability can be reduced by testing with
additional pairs when the test succeeds on the first five pairs, which is rare.
Thus, the total data complexity (plaintexts and ciphertexts) is
\begin{equation}
    D = 2 \cdot 2^8 \cdot 5 \approx 2^{11.32}.
\end{equation}

For the computational complexity, we need to test \(2^{24}\) keys for each set
of plaintexts as we only need to set \(k_{0, 1} = k_{0, 0} \oplus i\), since
\(k_{0, 0}, i \in \bF_q\). For each key, we will have \(2 \cdot 4\) S-box
lookups for 5 pairs to check \eqref{eq:aes-5-rounds-check}, giving a total
complexity of \(2^{24} \cdot 2 \cdot 4 \cdot 5 \cdot 2^8 = 2^{37.3}\), which
cooresponds to approximately\(2^{31}\) 5-rounds of AES (assuming 80 S-box
lookups per round).

Since the adversary knows \(k_0\), they can make a pair of words \(a_0^\prime,
b_0^\prime \in \bF_q^4\) that differ only in their first byte. The actual
plaintext pair is obtained by performing \(AK^{-1} \circ SB^{-1} \circ MC^{-1}\)
on it to obtain \(a_0, b_0\), which is used to create plaintexts \(p^0 =
\brak{a_0, 0, 0, 0}\) and \(p^1 = \brak{b_0, 0, 0, 0}\). However, this pair is
useless in recovering the other subkeys since the last three words are equal.
Instead, the yoyo can be used from this initial pair to generate pairs
\(\brak{p^{\prime 0}, p^{\prime 1}}\) that are with high probability different
in the last three words and whose difference after \(SR \circ MC \circ SB \circ
AK\) is non-zero only in the first word. To attack \(k_1\), notice that each of
the \(m\) pairs returned by the yoyo satisfy
\begin{equation}
    B\brak{p_1^{\prime 0} \oplus k_1} \oplus B\brak{p_1^{\prime 1} \oplus k_1} = \brak{0, w, 0, 0}
    \label{eq:aes-round-2-rel}
\end{equation}
for some \(w \in \bF_q\) and fixed \(k_1\). This is because the \(i\)-th byte of
the \(i\)-th word can be nonzero before \(SR\). Notice that
\eqref{eq:aes-round-2-rel} can be written as
\begin{equation}
    M^{-1}\brak{0, w, 0, 0} = w \cdot M_2^{-1} = s^4\brak{p_1^{\prime 0} \oplus k_1} \oplus s^4\brak{p_1^{\prime 1} \oplus k_1}.
\end{equation}
where \(M_i^{-1}\) denotes the \(i\)-th column of \(M^{-1}\). This can be used
to solve for \(k_1\) on fixing any byte in \(k_1\). Hence, at most \(4 \cdot
2^8\) guesses are spent on getting the correct key. Similarly, \(k_2\) and
\(k_3\) can be found using analogous relationships with columns of \(M^{-1}\).

A similar procedure can be followed for later rounds. To recover the remaining 3
round subkeys at once, the adversary should test the solutions against 4
plaintext pairs to ensure a comfortable margin against false positives. Since
the initial pair is useless, 5 pairs are used to recover the full key. The key
recovery algorithm is shown in \autoref{alg:aes-5-rounds-key}.

\begin{algorithm}
    \caption{Key Recovery for Five Rounds of AES}
    \label{alg:aes-5-rounds-key}
    \begin{algorithmic}[1]
        \Ensure{Secret key \(k_0\)}
        \For{\(i\) from 0 to \(2^8 - 1\)}
            \State \(p^0 \gets 0\), \(p^1 \gets 0\)
            \State \(p_0^0 \gets \brak{0, i, 0, 0}\), \(p_0^1 \gets \brak{1, 1 
            \oplus i, 0, 0}\)
            \State \(\cS \gets \cbrak{\brak{p^0, p^1}}\)
            \While{\(\abs{\cS} < 5\)}
                \State \(c^0 \gets enc_k\brak{p^0, 5}\), \(c^1 \gets 
                enc_k\brak{p^1, 5}\)
                \State \(c^{\prime 0} \gets \textsc{SimpleSWAP}\brak{c^0,
                c^1}\), \(c^{\prime 1} \gets \textsc{SimpleSWAP}\brak{c^1,
                c^0}\)
                \State \(p^{\prime 0} \gets dec_k\brak{c^{\prime 0}, 5}\),
                \(p^{\prime 1} \gets dec_k\brak{c^{\prime 1}, 5}\)
                \State \(p^0 \gets \textsc{SimpleSWAP}\brak{p^{\prime 0},
                p^{\prime 1}}\), \(p^1 \gets \textsc{SimpleSWAP}\brak{p^{\prime
                1}, p^{\prime 0}}\)
                \State \(\cS \gets \cS \cup \cbrak{\brak{p^0, p^1}}\)
            \EndWhile
            \ForAll{\(2^{24}\) key candidates \(k_0\)}
                \ForAll{\(\brak{p^0, p^1} \in \cS\)}
                    \If{\(l_3\brak{s^4\brak{p_0^0 \oplus k_0} \oplus
                    s^4\brak{p_0^1 \oplus k_0}} \ne 0\)}
                        \State Break and jump to next key
                    \EndIf
                \EndFor
                \State \Return \(k_0\)
            \EndFor
        \EndFor
    \end{algorithmic}
\end{algorithm}

\end{document}
