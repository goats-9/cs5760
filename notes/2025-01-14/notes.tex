% To familiarize yourself with this template, the body contains
% some examples of its use.  Look them over.  Then you can
% run LaTeX on this file.  After you have LaTeXed this file then
% you can look over the result either by printing it out with
% dvips or using xdvi.
%

\documentclass[twoside]{article}
%\usepackage{soul}
\usepackage{./lecnotes_macros}


\begin{document}
%FILL IN THE RIGHT INFO.
%\lecture{**LECTURE-NUMBER**}{**DATE**}{**LECTURERS**}{**SCRIBE**}
\lecture{1}{Differential Cryptanalysis}{Maria Francis and M. V. Panduranga Rao}{Gautam Singh}{14 January 2025}
%\footnotetext{These notes are partially based on those of Nigel Mansell.}

%All figures are to be placed in a separate folder named ``images''

% **** YOUR NOTES GO HERE:

\section{Differential Cryptanalysis}
In differential cryptanalysis, we exploit the XOR between plaintext pairs in a
chosen plaintext attack to find the secret key bits used in the cryptosystem.
Per DES round, the XOR of a pair of inputs remains invariant.

\begin{enumerate}
    \item Expansion \(E\) to get \(Si_E\).
    \item XOR with subkey \(K\) to get \(Si_I = Si_E \oplus Si_K\).
    \item Permutation after S boxes to get \(Si_O\).
\end{enumerate}

\begin{figure}[!ht]
    \centering
    \includegraphics[width=0.5\linewidth]{images/des_f.png}
    \caption{The \(F\) function of DES}
    \label{fig:des-f}
\end{figure}

The authors want to carry out a probabilistic analysis on the S boxes; this will
estimate the number of plaintexts to be used for cryptanalysis. Immediately,
they abstract out the key schedule by assuming all round subkeys are independent
of each other.

\section{Probability Analysis of S Boxes}
We now focus on the S boxes. Suppose \(Si_I^\prime = Si_I \oplus Si_I^*\) is the
input XOR and \(Si_O^\prime = Si_O \oplus Si_O^*\) is the output XOR. The
authors compute the joint probability distribution function (PDF) of the
input-output XOR pair \((Si_I^\prime, Si_O^\prime)\). This results in a 64-by-16
table of frequency distributions, since S boxes take 6-bit inputs to 4-bit
outputs. This is called the \emph{pairs XOR distribution table}.

For cryptanalysis, we can now talk about the conditional probability
\(\Pr(Si_I^\prime \mid Si_O^\prime)\). At each round of DES, we know
\(Si_I^\prime = Si_E^\prime\) and we observe \(Si_O^\prime\). Using the pairs
XOR distribution of that S box, we can get more information about \(Si_I\) and
\(Si_I^*\), which can then be used to compute \(Si_K = Si_I \oplus Si_E = Si_I^*
\oplus Si_E^*\). It is trivial to now extend this pairs XOR probability to the
entire \(F\)-function used in the Feistel network, since the only nonlinearlity
in the input XOR is provided by the S boxes.

S box \(Si\). Then, \(S_I\prime \rightarrow S_O^\prime\) with probability \(P =
\prod_{i=1}^8p_i\) by the F function of that particular round. That is, the
probabilities are \emph{multiplicative}. Further, the \(P_j\)'s across the
rounds \(j\) will also multiply, thereby reducing the chances of a particular
input-output XOR pair of the entire iterated cryptosystem. The aim is to keep
the total \(P = \prod_{i=1}^nP_i\) over the \(n\) rounds of this cryptosystem
high enough to be able to perform cryptanalysis quickly.

\section{Characteristic}

To formalize the cryptanalysis of iterated cryptosystems, we are introduced to
the notion of a \emph{characteristic}. This is a tuple \(\Omega = (\Omega_P,
\Omega_\Lambda, \Omega_T)\) which characterizes the XOR of a pair of plaintexts
as they go through the same cryptosystem. The authors write \(\Omega_\Lambda =
(\Lambda_1, \ldots, \Lambda_n)\) such that \(\Lambda_i = (\lambda_I^i,
\lambda_O^i)\) where \(\lambda_I^i\) is the input to the \(F\) function in the
\(i\)-th round and \(\lambda_O^i\) is the corresponding output. For intermediate
rounds, we have

\begin{equation}
    \lambda_O^i = \lambda_I^{i+1} \oplus \lambda_I^{i-1}.
    \label{eq:lambda-io-rel}
\end{equation}

\begin{figure}[!ht]
    \centering
    \includegraphics[width=0.5\linewidth]{images/des_char.png}
    \caption{An example of an \(n=2\) round characteristic.}
    \label{fig:des-char}
\end{figure}

Call a pair of plaintexts \emph{right} with respect to \(\Omega\) and an
independent key \(K\) if this pair generates the XORs using the key \(K\) over
the \(n\) rounds of encryption, and \emph{wrong} otherwise. The probability of a
characteristic is then the fraction of such plaintext pairs which satisfy
\(P^\prime = \Omega_P\). It is not hard to see that they can be concatenated and
the probabilities are multiplicative.

When \(\Omega_T\) is the swapped value of the halves of \(\Omega_P\), we get an
iterative characteristic: one that we can concatenate with itself to get
characteristics of arbitrary length. An example of an iterative characteristic
is shown in \autoref{fig:des-iter-char}. The best iterative characteristic has
probability about \(\frac{1}{234}\), where \(\psi = \texttt{19 60 00 00}\).

\begin{figure}[!ht]
    \centering
    \includegraphics[width=0.5\linewidth]{images/des_iter_char.png}
    \caption{Example of an iterative characteristic where \(\psi \rightarrow 0\).}
    \label{fig:des-iter-char}
\end{figure}

\section{Signal-to-Noise Ratio}
analyze a large enough number of plaintext pairs and find some bits of a subkey
With the knowledge of characteristics and their probabilities, we need to
(and thus the master key) entering certain S boxes. This is done with a simple
counting approach, and intuitively the most frequently suggested subkey is
likely to be the actual subkey used.

approaches. To do this, we define the \emph{signal-to-noise ratio} of a counting
However, we did not account for the time and memory complexity of such counting
scheme as the ratio of the number of right pairs to the average count, denoted
by \(S/N\). Suppose we want to find \(k\) subkey bits in a cryptosystem of block
size \(m\). Let \(\alpha\) be the average count per counted pair and \(\beta\)
be the fraction of counted pairs. Then, for a characteristic probability \(p\),

\begin{equation}
    S/N = \frac{mp}{\frac{m\alpha\beta}{2^k}} = \frac{2^kp}{\alpha\beta}.
\end{equation}

This shows that \(S/N\) is independent of the block size and the number of pairs
used. A larger \(S/N\) implies fewer plaintext pairs are needed and conversely.
To improve \(S/N\), we consider fewer plaintexts that can pair up to form right
pairs for various characteristics such as \emph{quartet} (four plaintexts with
two pairs each of two characteristics) and \emph{octet} (eight plaintexts, four
pairs each of three characteristics). This also saves memory since we can reuse
the plaintexts for different characteristics.

\end{document}
