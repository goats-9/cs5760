% To familiarize yourself with this template, the body contains
% some examples of its use.  Look them over.  Then you can
% run LaTeX on this file.  After you have LaTeXed this file then
% you can look over the result either by printing it out with
% dvips or using xdvi.
%

\documentclass[twoside]{article}
%\usepackage{soul}
\usepackage{./lecnotes_macros}


\begin{document}
%FILL IN THE RIGHT INFO.
%\lecture{**LECTURE-NUMBER**}{**DATE**}{**LECTURERS**}{**SCRIBE**}
\lecture{8}{Introduction to Gr\"{o}bner Bases}{Maria Francis}{Gautam Singh}{24 July 2025}
%\footnotetext{These notes are partially based on those of Nigel Mansell.}

%All figures are to be placed in a separate folder named ``images''

% **** YOUR NOTES GO HERE:

Gr\"{o}bner bases are a powerful tool for solving nonlinear polynomial systems.
They generalize the concept of a basis in linear algebra to polynomial ideals.
In cryptography, it was mainly used to compute signatures, namely the F5
signature scheme.

\section{Introduction}
\label{sec:intro}

\begin{definition}[Monomial]
    A monomial in \(k\sbrak{x_1, x_2, \ldots, x_n}\) is an expression of the
    form \(x_1^{a_1} x_2^{a_2} \cdots x_n^{a_n}\) where \(a_i\) are non-negative
    integers.
\end{definition}
The \emph{degree} of a monomial \(x_1^{a_1} x_2^{a_2} \cdots x_n^{a_n}\) is
defined as the sum of the exponents, i.e., \(a_1 + a_2 + \cdots + a_n\). A
shorthand notation for a monomial is \(x^a\) where \(a = \brak{a_1, a_2, \ldots,
a_n}\). In particular, \(x^{\alpha} = 1\) when \(\alpha = \brak{0, 0, \ldots,
0}\).

\begin{definition}[Polynomial]
    A polynomial \(f\) in \(k\sbrak{x_1, x_2, \ldots, x_n}\) with coefficients
    in \(k\) is a finite linear combination of monomials. Mathematically, a
    polynomial can be represented as \(f = \sum_{\alpha} a_{\alpha} x^{\alpha}\)
    where \(a_{\alpha} \in k\) and the sum is taken over a finite set of
    \(n\)-tuples \(\alpha = \brak{a_1, a_2, \ldots, a_n} \in \bN^n\).
\end{definition}
In particular, \(a_{\alpha}\) is called the \emph{coefficient} of the monomial
\(x^{\alpha}\) and the product \(a_{\alpha} x^{\alpha}\) is called a
\emph{term}.

Polynomials are closed in the ring \(k\sbrak{x_1, x_2, \ldots, x_n}\) under
addition and multiplication. However, in the usual sense, multiplicative
inverses of polynomials do not exist.

\begin{definition}[Affine Space]
    Given a field \(k\) and \(n \in \bN\), the \(n\)-dimensional affine space
    over \(k\) is given by
    \begin{equation}
        k^n \triangleq \cbrak{\brak{a_1, a_2, \ldots, a_n} \mid a_i \in k\ \forall\ i = 1, 2, \ldots, n}.
        \label{eq:affine-space}
    \end{equation}
\end{definition}
Affine spaces connect algebra with geometry.

\begin{proposition}
    Let \(k\) be an infinite field and \(f \in k\sbrak{x_1, x_2, \ldots, x_n}\)
    be a polynomial. Then, \(f\) is a zero function iff \(f\) is the zero
    polynomial.
\end{proposition}
Notice that this may not be the case in finite fields such as \(\bF_2\).

\begin{corollary}
    Let \(k\) be an infinite field and \(f, g \in k\sbrak{x_1, x_2, \ldots,
    x_n}\). Then, \(f = g\) in \(k\sbrak{x_1, x_2, \ldots, x_n}\) iff \(f\) and
    \(g\) are the same function.
\end{corollary}

\begin{theorem}[The Fundamental Theorem of Algebra]
    Every non-constant \(f \in \bC\sbrak{x}\) has a root in \(\bC\).
\end{theorem}
Polynomials that have roots in their field of coefficients are called
\emph{algebraically closed}.

\begin{definition}[Affine Varieties]
    Let \(k\) be a field and \(f_1, f_2, \ldots, f_s \in k\sbrak{x_1, x_2,
    \ldots, x_n}\). Then, the affine variety defined by \(f_1, f_2, \ldots,
    f_s\) is defined as
    \begin{equation}
        V\brak{f_1, f_2, \ldots, f_s} \triangleq \cbrak{\brak{a_1, a_2, \ldots, a_n} \in k^n \mid f_i\brak{a_1, a_2, \ldots, a_n} = 0\ \forall\ i = 1, 2, \ldots, s}
    \end{equation}
\end{definition}
In other words, the affine variety is the set of solutions to the polynomial
system defined by \(f_1, f_2, \ldots, f_s\). In particular, this reduces to
Gaussian Elimination when considering linear polynomials.

\begin{definition}[Ideal]
    A subring \(I \subseteq k\sbrak{x_1, x_2, \ldots, x_n}\) is called an
    \emph{ideal} if it satisfies the following properties.
    \begin{enumerate}
        \item (Absorption) If \(f \in I\) and \(h \in k\sbrak{x_1, x_2, \ldots,
        x_n}\), then \(h f \in I\).
        \item (Closure) If \(f, g \in I\), then \(f + g \in I\).
    \end{enumerate}
\end{definition}

\begin{definition}
    Let \(f_1, f_2, \ldots, f_s \in k\sbrak{x_1, x_2, \ldots, x_n}\). Then, the
    ideal generated by \(f_1, f_2, \ldots, f_s\) is given by
    \begin{equation}
        \langle f_1, f_2, \ldots, f_s \rangle \triangleq \cbrak*{\sum_{i=1}^{s}h_i f_i \mid h_i \in k\sbrak{x_1, x_2, \ldots, x_n}\ \forall\ i = 1, 2, \ldots, s}.
        \label{eq:ideal-gen}
    \end{equation}
\end{definition}
An ideal can have many generators. In particular, if \(\langle f_1, f_2, \ldots,
f_s \rangle = \langle g_1, g_2, \ldots, g_t \rangle\), then \(V\brak{f_1, f_2,
\ldots, f_s} = V\brak{g_1, g_2, \ldots, g_t}\).

We can also have the notion of an ideal given a variety space. That is, the
ideal of a variety \(V\) is defined as
\begin{equation}
    I\brak{\cV} \triangleq \cbrak*{f \in k\sbrak{x_1, x_2, \ldots, x_n} \mid f\brak{a_1, a_2, \ldots, a_n} = 0\ \forall\ \brak{a_1, a_2, \ldots, a_n} \in \cV}.
    \label{eq:ideal-variety}
\end{equation}
Note that for a given ideal \(J\), we have \(J \subseteq I\brak{V\brak{J}}\).

\end{document}
