% To familiarize yourself with this template, the body contains
% some examples of its use.  Look them over.  Then you can
% run LaTeX on this file.  After you have LaTeXed this file then
% you can look over the result either by printing it out with
% dvips or using xdvi.
%

\documentclass[twoside]{article}
%\usepackage{soul}
\usepackage{./lecnotes_macros}


\begin{document}
%FILL IN THE RIGHT INFO.
%\lecture{**LECTURE-NUMBER**}{**DATE**}{**LECTURERS**}{**SCRIBE**}
\lecture{2}{19 January 2025}{Maria Francis and M. V. Panduranga Rao}{Gautam Singh}
%\footnotetext{These notes are partially based on those of Nigel Mansell.}

%All figures are to be placed in a separate folder named ``images''

% **** YOUR NOTES GO HERE:

\section{Finding the Round Subkey of the F Function}

Throughout these notes, we make repeated use of the following lemma.

\begin{lemma}
    \label{lem:Fk}
    Suppose that \(F_k\brak{x} = X\), where \(x\) denotes the XOR of the inputs,
    \(X\) denotes the XOR of the outputs and \(k\) denotes the subkey used in
    \(F\). Then, given \(x\) and \(X\), we can extract \(k\) efficiently.
\end{lemma}
\begin{proof}
    Since we know \(x\), we know \(S_E\) and thus we know \(S_I\). Now, we also
    know  
\end{proof}

\section{Cryptanalysis of DES Reduced to 4 Rounds}

\end{document}
